\vspace*{-.5cm}

Na medida em que os sistemas computacionais se tornam mais pervasivos, a demanda por métodos de desenvolvimento rigorosos e composicionais cresce dramaticamente. No desenvolvimento baseado em componentes (CB-MDD), sistemas complexos (muitas vezes humanamente intangíveis) são construídos a partir de elementos mais simples, chamados componentes. Para atingir os objetivos desta abordagem na direção de torná-la uma disciplina formal de desenvolvimento, componentes e regras de composição devem ser formalizados. Além disso, considerando que os requisitos de um sistema estão em constante evolução, necessitamos de mecanismos para refinar e estender de forma confiável tais sistemas. O modelo de componentes BRIC formaliza os conceitos chave da abordagem CB-MDD, além de garantir corretude por construção se baseando em regras de composição que preservam propriedades comportamentais. BRIC, porém, por não possuir relações de extensão, não suporta evolução de modelos baseados em componentes.

Neste trabalho propomos relações de herança e refinamento para BRIC. Definimos uma semântica congruente que considera tanto a estrutura quanto o comportamento de componentes. Definimos refinamento como uma relação de pré-ordem, a qual é monotônica em relação as regras de composição de BRIC. Estendemos este modelo de componentes com suporte a extensibilidade via herança. As relações propostas permitem extensão de funcionalidade, ao mesmo tempo em que preservam conformidade de serviços, a qual é definida em termos de uma noção de convergência. Estabelecemos também uma conexão algébrica entre extensibilidade de componentes e refinamento. Até onde estamos cientes, este trabalho é pioneiro no desenvolvimento de noções de herança de componentes para uma abordagem CB-MDD formal e consistente. 

Também integramos o paradigma orientado a aspectos em BRIC. Contribuímos com uma abordagem para capturar, especificar e adotar aspectos no desenvolvimento confiável de sistemas baseados em componentes. Estabelecemos que componentes estendidos por aspectos preservam convergência, o que garante conformidade de serviços. Além disso, desenvolvemos uma conexão entre herança e aspectos, apresentando herança como um mecanismo para definir famílias de componentes e aspectos para capturar conceitos ortogonais sobre as mesmas.

Ilustramos a relevância prática das relações propostas através de três estudos de caso. No primeiro, modelamos um sistema autônomo de cuidados médicos, estendido pela adição de novas funcionalidades via herança e pela modularização de conceitos transversais de forma reusável e manutenível via aspectos. Na sequência, modelamos um sistema bancário, cujas funcionalidades são progressivamente implementadas e estendidas pelo uso de herança e refinamento. Finalmente, modelamos um sistema P2P cujo tráfico é reduzido por extensão via herança.

\begin{keywords}
herança de componentes. refinamento de componentes. correção por construção. \textit{design} orientado a aspectos para modelos de componentes. convergência comportamental. CSP
\end{keywords}
